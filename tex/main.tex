\documentclass{article}

\usepackage[a4paper,margin=1in]{geometry}

\usepackage{mystyle}
\usepackage[sort,comma,numbers]{natbib}

\title{Storylines}

\begin{document}
\maketitle

\section{Introduction}
% Improve coherency of generative models of text by discovering a notion of storylines.
Our goal is to improve the coherency of generative story models by modeling storylines.
As an alternative to black-box models with little explicit structure,
prior work has noted that including structure in the generative model improves coherence
in the generations \citep{yao2018storyline,fan2019structure}.

However, prior investigations into narrative structure used hand-crafted representations,
such as entity coreference or keywords.
We hope to alleviate the need for hand-crafted structure via structured generative models.

\section{Storyline Discovery with SBERT}
We first investigate whether we can recover structure that resembles a storyline
using pretrained models.
In particular, we use SBERT \citep{reimers2019sbert}
to map sentences $\bx_i \in \mcX^*$ to vector representations $\by_i \in \R^n$.
As each story consists of several sentences, we obtain a sequence of representations
corresponding to each sentence in a given story $\bY = \langle \by_0, \ldots, \by_T \rangle$.
We then compute an alignment between the sentences of two stories using DTW.

We evaluate the efficacy of SBERT and DTW without fine-tuning by determining whether
$DTW(Y_i, Y_j)$ correlates with an intuitive distance of stories.

Global

\section{Multisequence Alignment}
Although we found that the pairwise comparison using DTW correlated with intuitive distance,
the alignments themselves had issues.
We hope to extract robust storylines, common to multiple stories.
To accomplish this, we turn to multisequence alignment (MSA) which
considers multiple stories and produces a global alignment involving
all stories.

Global

\section{A Generative Model}
Both pairwise and multisequence alignment methods do not serve as good generative models,
instead focusing on structure discovery.


\section{Problem Statement}
We would like to learn a 


\newpage
\bibliographystyle{plainnat}
\bibliography{bib}

\end{document}
